\subsection{Некогерентное излучение}
Естественные источники света, как правило, являются некогерентными. Некогерентность в первую очередь означает, что величина поля в каждой точке является случайной величиной. Эти случайные величины не независимы и их связь определяет свойства некогерентности. К таким свойствам относятся пространственная и временная когерентность \cite{Ahmanov}.


\paragraph{Временная когерентность}
Временная когерентность описывается случайным характером амплитуды комплексного поля в зависимости от времени, что показано в формуле \ref{eq:TCField}, взятой из \cite{Ahmanov} и адаптированной для двумерного представления.
\begin{equation}\label{eq:TCField}
	E(x,y,t) = A(t)e^{+i{\omega}t}
\end{equation}
В этой формуле $E(x,y,t)$ - поле в некоторой плоскости, $\omega$ - частота этого поля, $A(t)$ - случайная комплексная амплитуда волны. Коэффициент корреляции $B_t(t_1,t_2)$ по времени показан в формуле \ref{eq:TCCorr}.
\begin{equation}\label{eq:TCCorr}
	B_t(t_1,t_2)=\frac
	{\left<\left(A(t_1) - \overline{A(t_1)}\right)\left(A^*(t_2) - \overline{A^*(t_2)}\right)\right>}
	{\sqrt{\left(A(t_1) - \overline{A(t_1)}\right)^2\left((A(t_2) - \overline{A(t_2)}\right)^2}}
\end{equation}
Числитель этого уравнения связан со спектральный представлением поля. Поэтому, временная когерентность непосредственно связана с монохроматичностью света.


\paragraph{Пространственная когерентность}\label{par:SpatialIncoherence}
Пространственная когерентность тоже имеет отношение к случайности комплексной амплитуды. Тогда, поле представимо по формуле \ref{eq:SCField}.
\begin{equation}\label{eq:SCField}
	E(x,y,t) = A(x,y)e^{+i{\omega}t}
\end{equation}
Наблюдаемая интенсивность этой волны рассчитывается по формуле \ref{eq:TCInt}. Коэффициент корреляции согласно \cite{Ahmanov} представлен на формуле \ref{eq:SCCorr}.
\begin{equation}\label{eq:TCInt}
	I(x,y) = \frac{c}{4\pi}\left<\left|E(x,y,t)\right|^2\right> = \frac{c}{4\pi}\left<\left|A(x,y)\right|^2\right>
\end{equation}
\begin{equation}\label{eq:SCCorr}
	B_t(x_1,y_1,x_2,y_2)=\frac
	{\left<\left(A(x_1,y_1) - \overline{A(x_1,y_1)}\right)\left(A^*(x_2,y_2) - \overline{A^*(x_2,y_2)}\right)\right>}
	{\sqrt{\left(A(x_1,y_1) - \overline{A(x_1,y_1)}\right)^2\left((A(x_2,y_2) - \overline{A(x_2,y_2)}\right)^2}}
\end{equation}
\par
Для полноценного представления случайной функции $A(x,y)$ необходимо представить её плотность вероятности. Вероятность обнаружить поле в точке $x_1,y_1$ с величиной от $u$ до $u+du$ отнесённое к величине $du$ - это одномерная плотность вероятности. Знание только этой зависимости недостаточно, так как эта плотность вероятности является маргинальной и не описывает совместное распределение полей в двух точках. Полная плотность вероятности будет бесконечно мерной величиной, в аргументе которой находятся все возможные пары точек $x_i,y_i$ для $\forall i = \overline{0 \dots \infty}$, смотреть уравнение \ref{eq:PropDens}.
\begin{equation}\label{eq:PropDens}
	\rho(x_1,y_1,u_1,x_2,y_2,u_2,\dots) = \lim\limits_{\Delta u_1, \Delta u_2, \dots \rightarrow 0}\frac
	{P\left(u_1 \le A(x_1,y_1) \le u_1 + \Delta u_1; \dots\right)}
	{\Delta u_1 \Delta u_2 \dots}
\end{equation}
Понятно, что такая функция очень сложна, поэтому введём приближения. В нулевом приближении $A(x,y)$ определяется только плотностью вероятности $\rho(x_1,y_1,u_1)$, а остальные выражаются рекурсивно по формуле \ref{eq:ReqProp}.
\begin{equation}\label{eq:ReqProp}
	\begin{cases}
		\rho(x_1,y_1,u_1,x_2,y_2,u_2) = \rho(x_1,y_1,u_1) \rho(x_2,y_2,u_2) \\
		\rho(x_1,y_1,u_1,x_2,y_2,u_2,x_3,y_3,u_3) =  \rho(x_1,y_1,u_1,x_2,y_2,u_2) \rho(x_3,y_3,u_3) \\
		\dots
	\end{cases}
\end{equation}
Первое выражение в системе \ref{eq:ReqProp} означает, что случайные величины в точках $x_1,y_1$ и $x_2,y_2$ независимы, что, как уже было сказано, никак не описывает корреляцию поля. Поэтому минимальным для описание некогерентности является первое приближение, в котором $A(x,y)$ описывается совместной функцией плотности вероятности $\rho(x_1,y_1,u_1,x_2,y_2,u_2)$. Остальные плотности вычисляются по той же схеме, что описана в системе \ref{eq:ReqProp}. Для того, чтобы получить маргинальное распределение в точке $x_1,y_1$, необходимо проинтегрировать по переменным $x_2,y_2,u_2$. Далее, по аналогии, можно вводить и другие приближения.