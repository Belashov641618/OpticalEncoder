\subsection{Некогерентное излучение}
Естественные источники света, как правило, являются некогерентными. Некогерентность в первую очередь означает, что величина поля в каждой точке является случайной величиной. Эти случайные величины не независимы и их связь определяет свойства некогерентности. К таким свойствам относятся пространственная и временная когерентность \cite{Ahmanov}.
\paragraph{Временная когерентность}
Временная когерентность описывается случайным характером амплитуды комплексного поля в зависимости от времени, что показано в формуле \ref{eq:TCField}, взятой из \cite{Ahmanov} и адаптированной для двумерного представления.
\begin{equation}\label{eq:TCField}
	E(x,y,t) = \frac{1}{2}\left(A(x,y)e^{+i{\omega}t} + A^*(x,y)e^{-i{\omega}t}\right)
\end{equation}