\nonumsection{Введение}
Цифровые технологии в настоящее время претерпевают значительные изменения, обусловленные быстрым развитием методов искусственного интеллекта, в том числе глубоких нейронных сетей. Эти методы находят широкое применение в самых различных областях, начиная от автоматического распознавания образов до обработки больших данных. Однако, вместе с растущими возможностями увеличиваются и требования к объему вычислительных ресурсов.
\par
Одним из перспективных направлений оптимизации алгоритмов машинного обучения является применение оптических технологий, которые могут существенно ускорить обработку данных благодаря свойствам света. В частности, значительный интерес представляет применение оптических методов для пред обработки изображений для нейронных сетей, что может кардинально снизить объемы необходимых вычислений при работе с нейронными сетями.
\par
Данная работа посвящена исследованию возможностей оптических методов пред обработки изображений в условиях пространственной некогерентности света. В качестве основной задачи выступает разработка и создание модели расчёта распространения некогерентного света через неоднородные структуры. Особенностью расчёта оптической системы данным методом заключается в возможности использования алгоритмов машинного обучения для оптимизации параметров. На основе данной модели была построена и обучена для классификации изображений гибридная оптоэлектронная нейронная сеть.