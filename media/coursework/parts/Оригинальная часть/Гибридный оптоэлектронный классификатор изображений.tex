\subsection{Гибридный оптоэлектронный классификатор изображений}
Идея создания гибридной оптоэлектронной сети для классификации основано на концепции, описанной в параграфе, посвящённом сжатия информации, в секции \ref{sec:OpticalCalcs}. Принципиальная схема представлена на рисунке \ref{ris:OEScheme}.
\begin{figure}[h]
	\centering{\includegraphics[width=1.0\linewidth]{figures/NetworkScheme.pdf}}
	\caption{Схема оптоэлектронной нейронной сети. $L$ - расстояние между слоями.}
	\label{ris:OEScheme}
\end{figure}
Излучение распространяется через модуляторы и попадает на матрицу детекторов. После регистрации интенсивностей детекторами, сигналы обрабатывается электронной нейронной сетью на компьютере. В качестве электронной модели была выбрана свёрточная архитектура ResNet18, основанная на остаточных блоках. Слои свёртки  имеют маленькие ядра, что положительно сказывается на скорости работы. Эта модель может обрабатывать изображения любой размерности, поэтому будет удобно сравнивать нейронную сеть без оптической пред обработки и с ней.

\paragraph{Параметры модели}
Параметры модели были выбраны эмпирическим путём исходя из корректности численных расчётов распространения света и скорости обучения модели. Длинна волны -- $500$ нм, размер расчётной области -- $5$ мм, количество узлов вычислительной сетки по одной оси -- $300$ шт, размер неоднородности маски -- $50$ мкм, расстояние между масками -- $90.5$ мм, количество некогерентных реализаций для усреднения -- $7$ шт. Детекторы модулировались функцией Гаусса с дисперсией -- $83$ мкм. Пространственная когерентность принимала различные значения.

\paragraph{Подготовка данных}
В качестве набора размеченных изображений для классификации был выбран CIFAR10. Этот набор состоит из $10$-ти классов, по $6000$ изображений размером $32 \times 32$ пикселя в каждом классе. Примеры изображений представлены на рисунке \ref{ris:CIFAR10}.
\begin{figure}[h]
	\centering{\includegraphics[width=1.0\linewidth]{figures/CIFAR10.png}}
	\caption{Примеры изображений набора CIFAR10.}
	\label{ris:CIFAR10}
\end{figure}

